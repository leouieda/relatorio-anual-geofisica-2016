\documentclass[12pt,a4paper]{article}
\usepackage[utf8]{inputenc}
\usepackage[T3,T1]{fontenc}
\usepackage[portuges]{babel}
\usepackage[noenc]{tipa}
\usepackage{tipx}
\usepackage[geometry,weather,misc,clock]{ifsym}
\usepackage{natbib}
\usepackage{hyperref}
\usepackage{pifont}
\usepackage{eurosym}
\usepackage{amsmath}
\usepackage{wasysym}
\usepackage{amssymb,amsfonts,textcomp}
\usepackage{color}
\usepackage{array}
\usepackage{hhline}
\usepackage{booktabs}
\usepackage{hyperref}
\hypersetup{
    pdftex,
    colorlinks=true,
    linkcolor=blue,
    citecolor=blue,
    filecolor=blue,
    urlcolor=blue,
    pdftitle={Relatorio Anual 2016 - Leonardo Uieda},
    pdfauthor={Leonardo Uieda},
}
\usepackage{graphicx}
\usepackage{pdfpages}

% Page layout (geometry)
\setlength\voffset{-1in}
\setlength\hoffset{-1in}
\setlength\topmargin{0.7874in}
\setlength\oddsidemargin{0.9846in}
\setlength\textheight{10.118099in}
\setlength\textwidth{6.2988997in}
\setlength\footskip{0.0cm}
\setlength\headheight{0cm}
\setlength\headsep{0cm}

% Pages styles
\makeatletter
\newcommand\ps@Standard{
  \renewcommand\@oddhead{}
  \renewcommand\@evenhead{}
  \renewcommand\@oddfoot{}
  \renewcommand\@evenfoot{}
  \renewcommand\thepage{\arabic{page}}
}
\makeatother
\pagestyle{Standard}
\setlength\tabcolsep{1mm}
\renewcommand\arraystretch{1.3}


\begin{document}

\begin{tabular}{lcr}
    \includegraphics{img/logo-on.jpg}
    & \hspace{5cm} &
    \includegraphics{img/logo-mcti.png}
\end{tabular}

\vspace{0.5cm}

\begin{center}
    \textbf{\LARGE RELATÓRIO 2016}

    \vspace{1cm}

    \textbf{MODELAGEM E INVERSÃO DE CAMPOS GRAVITACIONAIS EM COORDENADAS
            ESFÉRICAS}
\end{center}

\vspace{1.5cm}

\begin{flushleft}

{\bfseries
NOME DO ALUNO:
}
Leonardo Uieda

{\bfseries
NIVEL:
}
Doutorado

{\bfseries
ENDEREÇO RESIDENCIAL:
}
Rua Aguiar, 49 ap 104. 20261-120 Tijuca, Rio de Janeiro - RJ

{\bfseries
TELEFONES FIXO e CELULAR:
}
(21) 983636761

{\bfseries
EMAIL NÃO INSTITUCIONAL:
}
leouieda@gmail.com

{\bfseries
DATA DE INÍCIO NA PÓS-GRADUAÇÃO:
}
11/2011

{\bfseries
DATA DE INÍCIO DO PROJETO DE PESQUISA:
}
11/2011

{\bfseries
NOME DO ORIENTADOR:
}
Valéria C. F. Barbosa

{\bfseries
PERÍODO PREVISTO DE BOLSA DE ESTUDOS:
}
11/2011 - 10/2015

{\bfseries
PERÍODO A QUE SE REFERE O RELATÓRIO:
}
03/2015 - 02/2016

\vfill

{\small Este relatório foi entregue ao orientador no dia}
\rule{1.5cm}{0.4pt}
\hfill
\rule{4cm}{0.4pt}

\vspace{-0.2cm}

\hfill {\tiny ASSINATURA DO ORIENTADOR}

\vspace{0.5cm}

{\small O orientador emitiu parecer no dia}
\rule{1.5cm}{0.4pt}
\hfill
\rule{4cm}{0.4pt}

\vspace{-0.2cm}
\hfill {\tiny ASSINATURA DO ORIENTADOR}

\vspace{0.5cm}

{\small Este relatório e o parecer do orientador foi recebido na
SECRETARIA DA PÓS-GRADUAÇÃO no dia:}

\vspace{0.5cm}

\rule{3cm}{0.4pt}
\hfill
\rule{10cm}{0.4pt}

\vspace{-0.2cm}

\hfill {\tiny NOME E ASSINATURA DO FUNCIONÁRIO DA SECRETARIA}

\end{flushleft}

\newpage

%-----------------------------------------------------------------------------
% PAGINA DE INFORMACOES CURRICULARES
%-----------------------------------------------------------------------------
\begin{center}
\textbf{\large INFORMAÇÕES CURRICULARES, CONTENDO OS SEGUINTES ITENS:}
\end{center}

\vspace{1cm}

\begin{flushleft}

\noindent (1) TOTAL DE CRÉDITOS CURSADOS EM DISCIPLINAS: 12

\bigskip

\noindent (2) LISTA DE TODAS AS DISCIPLINAS CURSADAS ATÉ O MOMENTO PELO ALUNO
E OS CONCEITOS OBTIDOS

\bigskip

Tópicos de interpretação de dados gravimétricos e magnéticos - A\\
Minicurso: ``Electromagnetic methods in applied geophysics'' - (curso sem
conceito)\\
Fenômenos críticos em geociências - A\\
Inversão em métodos potenciais - A\\
Sísmica aplicada (ênfase em exploração de petróleo e gás) - A

\bigskip

\noindent (3) SITUAÇÃO DO ALUNO QUANTO AOS CRÉDITOS OBTIDOS NOS SEMINÁRIOS
ANUAIS

\bigskip

Aprovado nos anos de 2012 a 2015.

\bigskip

\noindent (4) SITUAÇÃO DO ALUNO QUANTO AO EXAME DE PROFICIÊNCIA

\bigskip

Aprovado.

\bigskip

\noindent (5) SITUAÇÃO DO ALUNO DE DOUTORADO QUANTO AO EXAME DE QUALIFICAÇÃO

\bigskip

Aprovado no ano 2013.

\bigskip

\noindent (6) LISTA DAS REUNIÕES CIENTÍFICAS EM QUE PARTICIPOU NO PERÍODO A QUE
SE REFERE O RELATÓRIO, COM O TÍTULO E AUTORES DOS TRABALHOS APRESENTADOS

\bigskip

Nenhuma.

\bigskip

\noindent (7) LISTA DOS ARTIGOS PUBLICADOS, ACEITOS OU SUBMETIDOS

\bigskip


(Em revisão) Uieda, L., V. C. F. Barbosa, C. Braitenberg (2015),
Tesseroids: forward modeling gravitational fields in spherical coordinates,
Geophysics.

\bigskip

(Em revisão) Carlos, D. U., Uieda, L., and V. C. F. Barbosa (2015),
How two gravity-gradient inversion methods can be used to reveal different
geologic features of ore deposit - a case study from the Quadrilátero
Ferrífero (Brazil),
Journal of Applied Geophysics.

\bigskip

(Publicado) Oliveira Jr, V. C., D. P. Sales, V. C. F. Barbosa, and L. Uieda
(2015), Estimation of the total magnetization direction of approximately
spherical bodies, Nonlinear Processes in Geophysics.

\bigskip

\noindent (8) OUTRAS ATIVIDADES RELEVANTES NO PERÍODO (PARTICIPAÇÃO EM
TRABALHOS DE CAMPO, ESCOLAS ETC)

\bigskip

Ministrei o curso "Python como uma ferramenta numérica em ciências da Terra:
uma nova abordagem de programação", oferecido durante a XVIII Escola de Verão
de Geofísica do IAG - USP, com duração de 20h.  O material didático e
informações sobre o curso estão disponíveis em
\url{http://www.leouieda.com/teaching/python-iag-2016.html}


\bigskip

\end{flushleft}

\vspace{4cm}

%-----------------------------------------------------------------------------
% PROJETO
%-----------------------------------------------------------------------------
\begin{center}\textbf{\large PROJETO ORIGINAL}\end{center}

\vspace{1cm}

\textit{
Nesta parte do relatório o estudante deve incluir o projeto de pesquisa tal
como apresentado à Comissão de Pós-Graduação em Geofísica do Observatório
Nacional na época da inscrição.
}

\includepdf[pages={-},offset=2.5cm -2.5cm]{projeto.pdf}


\newpage

%-----------------------------------------------------------------------------
% DESENVOLVIMENTO DO PROJETO DE PESQUISA NO ÚLTIMO RELATÓRIO
%-----------------------------------------------------------------------------
\begin{center}
\textbf{\large RELATO DO DESENVOLVIMENTO DO PROJETO DE PESQUISA NO ÚLTIMO
RELATÓRIO}
\end{center}

\vspace{1cm}

No período entre Março de 2014 e Fevereiro de 2015,
eu finalizei os testes e resultados referentes a modelagem direta com
tesseroides.
Também escrevi e submeti um artigo com esses resultados para a revista
Geophysics.
No final do período, iniciei o desenvolvimento da metodologia de inversão
não-linear para mapear a Moho utilizando dados de gravidade.


\vspace{3cm}


%-----------------------------------------------------------------------------
% TRABALHO DE PESQUISA DESENVOLVIDO NO PERÍODO
%-----------------------------------------------------------------------------
\begin{center}
\textbf{\large DESCRIÇÃO DETALHADA DO TRABALHO DE PESQUISA DESENVOLVIDO NO
PERÍODO DO RELATÓRIO}
\end{center}

\vspace{1cm}

{\centering\bfseries\itshape
Metodologia aplicada ou desenvolvida
e
Resultados parciais já obtidos
\par}

\bigskip

No período entre Março de 2015 e Fevereiro de 2016,
eu finalizei, submeti e efetuei a primeira rodada de revisões do artigo
"Tesseroids: forward modeling gravitational fields in spherical coordinates"
(primeira parte da minha tese).
Nesse mesmo período, desenvolvi um método de inversão não-linear para mapear o
relevo da descontinuidade de Mohorovičić (Moho) a partir de dados de gravidade.
Fiz a implementação computacional do método proposto, testes com dados
sintéticos e uma aplicação para estimar a profundidade da Moho para a
América do Sul.
Por fim, escrevi um artigo (a segunda parte da tese) relatando o método
proposto e os resultados obtidos para submissão para revista Geophysical
Journal International.
Este manuscrito está anexado abaixo.


\bigskip

{\centering\bfseries\itshape
Dificuldades encontradas e como elas estão sendo superadas
\par}

\bigskip

A maior dificuldade encontrada foi o curto tempo disponível para a realização
do trabalho. Fomos capazes de obter os resultados almejados e escrever o
artigo correspondente com atraso de aproximadamente 4 meses do tempo previsto
no último relatório.
Atualmente, o artigo está sendo revisado por minha orientadora antes da
submissão.

\bigskip
\bigskip

{\centering\bfseries\itshape
Manuscrito do artigo sobre inversão não-linear para mapeamento da Moho.
\par}


\input{parte-paper.tex}

\bibliographystyle{gji}
\bibliography{references}

\newpage

%-----------------------------------------------------------------------------
% PRÓXIMAS ETAPAS DO TRABALHO DE PESQUISA
%-----------------------------------------------------------------------------
\begin{center}
\textbf{\large PRÓXIMAS ETAPAS DO TRABALHO DE PESQUISA}
\end{center}

\vspace{0.5cm}

\begin{flushleft}

\textbf{Atividades de pesquisa previstas para o próximo período:}

\begin{enumerate}
    \item Submissão do segundo artigo para a revista Geophysical Journal
        International.
    \item Elaboração da tese baseada nos dois artigos.
    \item Defesa.
\end{enumerate}

\textbf{Atividades acadêmicas previstas para o próximo período:}

\bigskip

Nenhuma.

\bigskip

\textbf{Cronograma detalhado das atividades:}

\begin{center}
\begin{tabular}{l|l}
    \toprule
    \textbf{Atividade} & \textbf{Período de execução} \\
    \midrule
    Submissão do artigo & Março\\
    Finalização da tese & Março\\
    Defesa & Março-Abril\\
    \bottomrule
\end{tabular}
\end{center}


\bigskip

\textbf{Data prevista de conclusão do mestrado ou doutorado:} Março-Abril/2016

\end{flushleft}


\end{document}
